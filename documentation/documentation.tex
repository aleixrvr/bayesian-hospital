


\documentclass[12pt]{article}

\usepackage{amsfonts}
\usepackage{amsmath}
\usepackage{pgf}
\usepackage{tikz}
\usepackage{dcolumn}
\usetikzlibrary{arrows,automata}
\usepackage[latin1]{inputenc}

\usepackage[round]{natbib}
\definecolor{ashgrey}{rgb}{0.7, 0.75, 0.71}

\def\bibsection{\section{References}}

\hfuzz=25pt
\parskip=12pt
\pagestyle{empty}

\setlength{\textwidth}{6.5in}
\setlength{\textheight}{8.8in}
\setlength{\oddsidemargin}{0in}
\setlength{\topmargin}{-0.5in}
\setlength{\jot}{0.3cm}

\usepackage{hyperref,graphicx,textcomp,color,enumitem,tikz}
\usetikzlibrary{shapes.geometric, arrows}
\newcommand\school{UPF}

\newcommand\opc[1]{{\color{red}#1}}

\begin{document}

\def\moveover{\hskip 4.2 true in}





%\vfill
\bigskip
% \settabs 3 \columns

%\moveover \today.\hfil\break
\bigskip


\title{Bayesian Hospital - Documentation}

\date{}

\maketitle

\subsection*{Team}

Barcelona GSE Data Science Center
Omiros Papaspiliopoulos, Aleix Ruiz de Villa, Reid Falconer, Max Zebhauser and Nandan Rao

\subsection*{Platform}

We have used BigQuery from Google Cloud Platform to manage and process MIMIC's data. Some links that may be useful

\begin{itemize}
	\item MIMIC's documentation: https://mimic.physionet.org/
	\item MIMIC's databse schema https://mit-lcp.github.io/mimic-schema-spy/
	\item Code repository: https://github.com/MIT-LCP/mimic-code
	\item Visualization tool: http://hdsl.uwaterloo.ca/visualization-tool/
\end{itemize}

\subsection*{Uploading data to Big Query}

The best way to upload data si following step by step the tutorial from MIMIC' github repository 
https://github.com/MIT-LCP/mimic-code/tree/master/buildmimic/bigquery

\subsection*{Connect to BigQuery via R}

For this you can use the 'BQ connection example.R' file. First time you use it, it asks you for your google authorization and saves a file in your disk to remember it. \textbf{Note}: If you use any type of code repository, be careful to ignore this file, otherwise you will upload your credentials to your repository.

\subsection*{Build Tables}


%TODO Reid

\subsection*{Data Exploration App}

For running this Shiny App, you just need to open and run the RStudio project in the folder 'exploration-app'.

\subsection*{Model Building}

For building the models that will be later on used by the outflow-app, open the bayesian-hospital RStudio project in the main folder and then execute:

\begin{itemize}
	\item 'model-building/data/final-data-retrieval.R' to save summarized data locally 
	\item 'model-building/build\_model.R' to build and save the models in 'model-building/model/'
\end{itemize} 


\subsection*{Outflow App}

%TODO Reid

\end{document}
